\documentclass[12pt]{article}

\usepackage[a4paper,lmargin=25pt,rmargin=25pt,bmargin=0.8cm,includefoot]{geometry}
\usepackage[T1]{fontenc}
\usepackage{parskip}
\usepackage{microtype}
\usepackage{marginnote}
\usepackage{fwlw}
\usepackage{librebaskerville}

\pagestyle{NextWordFoot}
\reversemarginpar
\renewcommand*\marginfont{\footnotesize}

\begin{document}

\section{Introduction}

\large{
To all but the dinosaurs, meteors are quite cool. As part of a year-long
project at Exeter Mathematics School, I have studied 6 different areas of radio
meteor detection, with the aim of analysing variation and phenomena associated
with this method of detection. Before I present these studies, we must consider
what a meteor {\it is} and the mechanisms behind meteor detection.
}

\section{What is a meteor?}

\large{
The title of this talk is why meteors rock, but I must admit that I have lied
to all of you: only half of meteors rock. The other half are metallic in
composition. It was a catchy title though.\\

Meteors have three stages to their lives. First, they exist as small rocky or
metallic bodies floating through space, where they are referred to as
meteoroids. When, or if, a meteoroid enters the atmosphere of Earth, it becomes
a meteor. As re-entry occurs, the meteor is heated intensely, owing to friction
with air molecules. The air around the meteor is also heated, to such an
extreme temperature that atoms making up the air are ionised, stripping
electrons away from them. The super-heated trail left by the meteor is what you
see as a `shooting star'. If the meteor makes it to the ground, it becomes a
meteorite.
}

\section{Meteor showers}

\large{
Typically, meteors are observed as part of `showers'. These are events, lasting
over a couple of weeks, where there is an increased number of meteoroids in the
space around Earth, resulting in more meteors, all of which appear to originate
from a single point in the sky, called the radiant. Meteors that are not part
of a shower are referred to as `sporadic' meteors.\\

Meteor showers occur throughout the year, and are each attributed to a comet,
or sometimes an asteroid, which causes the shower. As the comet orbits the Sun,
a trail of dust is left, producing a meteor `stream'. The Earth moving through
this stream is what constitutes a meteor shower. When Earth moves through the
most dense part of the stream, many more meteors are observed. This is known as
the `peak' of the shower.\\

The names for the showers are derived from the constellation where the radiant
lies. For example, the Perseids shower's radiant is in Perseus. Each shower
occurs at roughly the same time each year, since the trail left by the source
comet doesn't move relative to Earth, meaning it lies at the same point in
Earth's orbit each year, occurring at the same time. \\

These showers can vary widely. Some may produce lots of fireballs, which are
meteors that explode under the pressure of re-entry, lighting up the entire
sky.  Some may only produce very small meteors, since there is only small sized
debris left by a comet. This variation occurs from the meteor `stream'
spreading out as the trail orbits the Sun. As well as this, over successive
years of meteor showers without another orbit of a comet, the larger debris is
collected by Earth and only smaller debris is left. This can result in peak
years, when many more meteors are detected, after an orbit of the source comet.
Conversely, this can also cause the peak of a shower to vary in intensity over
a number of years.
}

\section{Radio Meteor Detection}

\large{
Meteors are most commonly observed visually, however radio meteor detection
offers an alternative. This method is possible because the ionised,
super-heated air left as a trail by a meteor is reflective to radio waves. The
process starts with a transmitting antenna, often a significant distance away
from those observing the meteors, which emits a radio signal into the
atmosphere. This signal must ideally spread across a wide area of sky, allowing
the best chance of detecting a meteor. The signal must also be at a suitable
angle such that the signal intercepts the meteors where most re-entries occur:
between 60 and 110 km.\\

When the signal intercepts the meteor trail, it is reflected and the signal can
then be received by an observer. The part of a received signal that indicates a
meteor detection is referred to as an `echo'. Often the transmitter will be
beyond the horizon of the receiver, allowing no direct signal, which is best
for meteor detection, since the received echo is clearer.\\

What can we do with a meteor echo? There are two main methods: simply counting
the number of detections an hour, or capturing images of echoes for further
analysis. Each of these methods depends on a precursor, where the received
signal is represented as a `waterfall' plot. These are plots that show
frequency and intensity over time. The frequency shown in the plot is not
necessarily accurate due to the way the radio signal is received. The
intensity, in dB, is represented using a colour scale. Meteor echoes have a
distinctive look in a waterfall plot. \\

Using this waterfall plot, we can complete the two methods. For counting the
number of detections an hour, we analyse the signal, and declare any peaks
above a certain intensity threshold a meteor. In order to capture images, we
generally use a 2D representation of the waterfall plot. \\
}

\section{Antenna Type}

\large{
In my analyses, I have used data from the Radio Meteor Observation Bulletin
organisation, known as RMOB, which provides a database of hourly counts from
observers around the world. The majority of observers contributing data are 
amateurs, and have various set-ups. \\

My first analysis focused on the difference in data collected by
observers with different types of antenna. Overall there are 9 clear antenna
types, with some miscellaneous types. Overwhelmingly the most popular is the
Yagi antenna, which you may recognise as what was once the most popular choice
of TV aerial.\\

My results show that there is little difference between
detection results collected by observers with very different antennas. This is
not surprising: if there was a clearly superior antenna, almost all observers
would use this.  
}

\section{Spatial Variation}

\large{
Along a similar vein, how does an observer's location influence their results?
In order to analyse the spatial variation, we study the change in a number of
data characteristics in relation to longitude and latitude. The short answer
is: no. I found that, generally, longitude and latitude do not appear to
influence the quality or quantity of an observer's data. However, it is
difficult to make a conclusive summary since there are only 4 or 5 observers at
some locations. \\

There are some interesting results to note, however. For both longitude and
latitude analyses, the skew of the data, indicating the shape of the
distribution, is positive. This makes sense, since there is a lower limit to 
the counts, 0, but no upper limit. \\

Another clear result is a correlation between longitude and peak hour of
diurnal shift. What is diurnal shift, you ask?
}

\section{Diurnal Shift}

\large{
Diurnal shift, also known as diurnal variation, is a phenomenon exhibiting an
increase and decrease in meteor detection counts over the course of a day. This
is a well documented observation, appearing to occur across the globe. Through
my own research I found no standardised model of what causes diurnal shift,
though there appears to be an accepted explanation for the event.\\

Based on existing explanations, I made my own model, an extension and more
mathematical description of that already given, which is based on the idea that
the overall intercept velocities of sporadic meteors changes due to the Earth's
orbital velocity. This is better illustrated with a diagram. \\

We must start with the assumptions that sporadic meteors have no preferred
direction to enter Earth's amtmosphere. We also assume that the number of
meteors detected is proportional to the intercept velocity of any given meteor.
Meteors are mostly only detected if they are travelling tangent to Earth's
atmosphere, which complicates things. However, all of these velocities will
cancel, leaving an overall incident velocity perpendicular to Earth's surface.
From the diagram we see that the incident velocity is the addition of a
component of Earth's orbital velocity, and the meteor's velocity. At sunset,
the meteor's incident velocity is added to Earth's orbital velocity, giving a
higher intercept velocity and thus more meteors are detected.\\ 

In order to test my model, I used the RMOB data to analyse the diurnal
variation of each observer. Considering the form of diurnal shift for different
locations, we see that the model fits well with the data: plots of diurnal
shift for observers in Europe, North America, as well as Asia \& Australia all
produce diurnal variation fitting a sinusoidal curve, as predicted by the
model. As of yet I have no explanation as to why some locations have a greater
mean detection count than others. It could be due to detection setups,
interference from other sources, or simply low samples sizes.\\

The model also predicts that the peak hour will be approximately 6am, though
only approximately since nature is unfortunately not as well behaved as we'd
like.  This map shows each observer as a dot, with the colour based on the peak
hour of their diurnal shift. It is somewhat clear that the peak hour is based
on their longitude. Correcting for each observer's timezone (based on
longitude, not {\it actual} timezones), we see that the majority of observers
have a peak hour of 6am, which agrees with the model.\\

Although rather hard to discern, my results also indicate that the peak hour of
diurnal shift does not change over time, nor the intensity. However, the
sinusoidal fit of an observer's diurnal variation appears to be less pronounced
between 2005 and 2011. This seems a strange result, but all will become clear.
}

\section{Temporal Variation}

\large{
In addition to analysing the change in data by location, we can analyse the
change through time, on three scales: variation since 2000, variation over a
year, and variation over the course of a month.\\ 

Regarding the month scale, there is no clear change in detection count for any
observer. There is a very slight increase, but only on the order of 4 or 5
counts - this is not significant, though the change is clear in each location
category. Very small increases such as this are most likely caused by a large
meteor shower occuring on the same day of the month each year, such as the
Perseids, which peak between the 12th and 14th of August each year.\\

I found that, on the scale of a single year, the detection counts are lower
from January until May. Beyond and including June, the detection counts are
larger by around 20. The peak of this increase is in June. Peculiarly this
variation doesn't appear to apply to the Asia \& Australia category. However,
there is still an increase of a similar magnitude, simply peaking two months
earlier.  This increase could be caused by several factors. It could be due to
the timings of meteor showers, or potentially the position of planets relative
to Earth. For example, Jupiter often `sweeps' up debris in space, preventing it
reaching Earth. However, this is unlikely, since the variation is periodic and,
although the positioning of planets is also periodic, it is with a very
different frequency. In order to investigate this further, factoring out the
background number of detection counts would be necessary, as well as analysing
the change in period of variation over time.\\

To my surprise, I found a significant change in detection counts since 2000.
Between 2005 and 2011, the detection counts increase up to almost 2 times the
normal value. Upon further research, I found this was a documented phenomenon.
Other studies have noted the correlation between the maximum of the detection
counts and minimum of the solar cycle. This is a reasonable link: when the Sun
is less active, less solar wind is generated, giving a lower electron density
in the upper atmosphere of Earth, which itself provides less interference to
radio signals, allowing a greater number of detections.\\

In addition to this, I have found that the Earth facing towards or away from
the Sun has no influence on how many meteors are detected. In addition to this,
the results suggest that the distribution of sporadic meteors is relatively
uniform in space, giving validity to assumptions made for the diurnal shift
model.\\
}

\section{Zenithal Hourly Rate}

\large{
It is important to have a standard way of measuring the activity of a meteor
shower, enabling predictions to be made by models of debris streams. This
measure is called the Zenithal Hourly Rate (ZHR). ZHR indicates how many
meteors an observer can expect to see in an hour if they can see the entire
sky, with perfect viewing conditions, and the radiant is directly overhead,
called the zenith. The formula for calculating the ZHR uses certain correction
factors to compensate for {\it not} having these conditions. For example, there
is a correction for the height of the radiant above the horizon, the field of
view, and the limiting magnitude, which is the dimmest magnitude an object you
can see.\\

I modified the formula so that it is applicable to radio meteor detection. The
correction factor for radiant height is inaccurate, indicating that no meteors
are seen when the radiant is below the horizon, which is not the case. It also
results in an infinite ZHR when the radiant approaches the horizon. I
formulated my own correction factor, and have used that instead. In order to
use the formula for radio meteor detection, other factors must be changed. For
starters, the field of view correction factor cannot be used since it is not
known exactly how much of the sky each antenna can see.\\

As well as this, the correction factor for limiting magnitude is invalid. This
factor uses the idea of population index, which indicates how many times more
meteors you expect to see with an increase in magnitude of 1. However, this
assumes that there is an exponential increase up to the visual limiting
magnitude of 6.5, which is valid, but this cannot possibly be true up to the
limiting magnitude of a radio antenna which is {\it much} greater. If it {\it
was} the case, we would expect to see trillions of meteors an hour -- and we
don't, luckily.\\

Using the formula for radio observation also requires a correction for the
background detection rate. This is not applicable to visual observation since
the background detection rate is close enough to 0 to be negligible. However,
even when there are no active showers, many meteors are detected by radio
observation. When these correction factors are included, the formula is like
so.  This comes with the unfortunate assumptions that the transmitting antenna
can see the entire sky.\\

Despite these assumptions, the results correlate well with records of visual
maximum ZHRs and expected shower ZHRs. There is a clear offset between the
results, which would most likely be accounted for by knowing the limiting
magnitude and field of view of each respective radio observation. The results
demonstrate that ZHR is an applicable measure of activity through radio
observation, though there are of course improvements to be made.\\
}

\section{Root Mean Square Difference Image Analysis}

\large{
As mentioned at the start of this talk, the main method of radio meteor
detection is to count the number of meteors detected in an hour. Also mentioned
is the alternative of recording images of waterfall plots. In an attempt to use
this as a quantitative measure of meteor flux (a fancy word for flow, i.e. the
number of meteors coming through Earth's atmosphere), I wrote a program that
uses Root Mean Square Difference.\\

The idea of using RMS Difference lies in how meteor echoes are represented in
waterfall plots. Since the most intense signals, indicating meteors, are white
in colour, which has the largest colour value when represented in an image. RMS
difference is the square root of the mean of the squared differences between
corresponding pixels in two images. Effectively, it's a measure of how
different two images are. By comparing a waterfall plot to a baseline image --
the choice of baseline is arbitrary -- we gain a value for how different two
images are.  Considering the change in this value over time indicates the
change in meteor flux.\\

Using separately recorded hourly counts, from the same observation station
(which is here, incidentally), we see that the correlation between
my own results and the detection counts are good. The correlation coefficient
of these two sets of data is 0.616, indicating a good positive correlation.
That is, if one increases, the other does so accordingly. This provides a new
way of analysing meteor detection results, which has no cost on top of
detection equipment and only requires one program to use. 
}

\section{Conclusion}

\large{
To conclude, more data is needed! A more uniform coverage of the globe, and
more data, would dispel much of the doubt over results with low sample sizes.
However, saying this, I believe I have found some interesting results, with a
new method of analysis, a new method of calculating ZHR, an extended model of
diurnal shift, and results on the variation of meteor characteristics from a
dataset of almost 350 observers. Of course, more investigation is always
needed. I hope you can now tell others why meteors rock.  
}

\end{document}
